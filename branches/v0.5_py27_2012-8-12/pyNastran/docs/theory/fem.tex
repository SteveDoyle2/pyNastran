%\pdfbookmark[1]{Static Solution 101}{sol101}
\section{Static Solution 101}
\label{sol101}
   \subsection{Global Stiffness Matrix Assembly Process}
     See the specific element you're interested in to find out how to calculate $[K_e]$ and $[K]$.
     Once $[K]$ is calculated, it can be inserted into the matrix $[K_{gg}]$.  The $[K_{gg}]$ matrix
     is the set of all equations defining element connectivity with no constraints.
     
     The constraints for the SPCs and MPCs are applied to the $[K_{gg}]$ matrix to form the  $[K_{aa}]$
     matrix.  Similarly loads are applied to form the $[F_g]$ vector, and modified when
     constraints are applied to form the $[F_a]$ vector. This is then solved:
     \[ {x} = [K_{aa}]^{-1} {F_a}  \]

   \subsection{SPC Constraints}
   \label{SPC Constraints}
     An SPC equation is of the form:
      \[ u_i =\delta u \]
     Typically, the value of $\delta u$ is 0, such that:
      \[ u_i = 0 \]
     This constraint may be in the global or local frame.  If it's in the global frame, life is easy.
     If it's in the local frame, 

   \subsection{MPC Constraints}
   \label{MPC Constraints}
     An MPC equation is of the form:
      \[ \Sigma_{i=0}^n A_i u_i =0 \]
     so an equation of:
      \[ A_1u_1 + A_2u_2 + A_3u_3 = 0 \]
     is rewritten as:
      \[ u_1 + \frac{A_2}{A_1}u_2 + \frac{A_3}{A_1}u_3 =0 \]
     This equation may now ``replace'' the original equation for $u_1$.
     
     Note that the MPC equation must be in the global coordinate frame.  See section \ref{SPC Constraints} for more detail.
     